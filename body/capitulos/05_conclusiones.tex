\chapter{Conclusiones y trabajo futuro}
\label{conclusiones}

 En este capítulo, se finalizará la memoria con las conclusiones del trabajo realizado, haciendo una valoración del mismo y de los resultados obtenidos del desarrollo. Además, se presentarán las vías de trabajo futuro que han ido apareciendo a medida que se realizaba el \gls{tfg}. En concreto, se expondrán dichas vías, y se concluirá con un breve análisis sobre la viabilidad de las mismas.
%%%%%%%%%%%%%%%%%%%%%%%%%%%%%%%%%%%%%%%%%%%%%%%%%%%%%%%%%%%%%%%%%%%%%%%%%%%%%%%%%%%%%%%%%%%%%%%%%
\section{Conclusiones del \glsentryshort{tfg}}

La finalidad de los casos de uso era la de ver con qué tecnología (\gls{xdp}, P4) es más sencillo y eficiente definir el \textit{datapath} de los dispositivos \gls{iot} en entornos \gls{sdn}. Por tanto, antes de realizar las conclusiones, se deberá revisar los resultados de las evaluaciones de ambas tecnologías en los distintos entornos, cableado e inalámbrico. \\
\par
Como se puede apreciar en la tabla \ref{tab:useCases}, con la tecnología \gls{xdp} en medios cableados no se pudo realizar un Broadcast de forma nativa, es decir, dicha tecnología no fue suficiente para lograr el broadcast. Es verdad que dicha limitación fue estudiada, y se planteó una solución, haciendo uso de un programa \gls{bpf} adicional en el \gls{tc}, pero aún teniendo en cuenta que se solucionó, no se puede afirmar que la tecnología lo soporte.  En cambio, si se observa dicha funcionalidad en un entorno Wireless, ya no habría dicha limitación dado que no sería necesario clonar los paquetes. Generalmente cambiando la dirección destino del paquete de capa dos a difusión, y transmitiéndolo al medio, será suficiente para lograr el Broadcast. \\
\par



En cambio, la tecnología P4, a diferencia de \gls{xdp}, tiene una interfaz de alto nivel para definir funcionalidades de broadcast/multicast. Esta interfaz, comúnmente conocida como ``grupos multicast", permite definir en una estructura de datos de tipo JSON por qué puertos inundar y con qué cantidad de réplicas por puerto. De forma adicional, cabe mencionar que los grupos multicast permiten modificar el tipo de instancia de los paquetes clonados, siendo posible diferenciar los paquetes originales de los clonados. \\
\par

\begin{table}[ht]
\centering
\begin{tabular}{|l|c|c|c|c|}
\hline
\rowcolor[HTML]{EFEFEF} 
\multicolumn{1}{|c|}{\cellcolor[HTML]{EFEFEF}\textbf{Caso de uso}} & \textbf{XDP} & \textbf{P4} & \textbf{P4-Wireless} & \textbf{XDP-Wireless} \\ \hline
case01 - Drop                                                      & \cmark       & \cmark      & \cmark               & \cmark                \\ \hline
case02 - Pass                                                      & \cmark       & \xmark      & \xmark               & \cmark                \\ \hline
case03 - Echo server                                               & \cmark       & \cmark      & \cmark               & \cmark                \\ \hline
case04 - Layer 3 forwarding                                        & \cmark       & \cmark      & \cmark               & \cmark                \\ \hline
case05 - Broadcast                                                 & \xmark       & \cmark      & \cmark               & \cmark                \\ \hline
\end{tabular}
\caption{Resumen sobre los casos de uso desarrollados}
\label{tab:useCases}
\end{table}

Volviendo a la tabla \ref{tab:useCases}, se quiere hacer especial hincapié en que la tecnología P4 no soporta la funcionalidad de dejar pasar o delegar los paquetes, sin afectarles el plano de datos programado en P4. Por otro lado, con \gls{xdp} siempre se puede pasar el paquete al Kernel para que se encargue él del procesamiento. Sin embargo, en P4, se debe definir de forma exclusiva todo el \textit{datapath}, por lo que no hay a quien delegar el paquete, es decir, se tiene que encargar el propio programa P4. Este punto es muy positivo para \gls{xdp}, ya que acciones sencillas y repetitivas pueden ser definidas en un programa \gls{xdp} y delegar el resto de funcionalidades al Kernel. Por tanto, se actuaría de forma cooperativa y ganando en rendimiento como se vio en el case04 (\ref{xdp_ether_case04}), donde se trabajaba de forma conjunta para obtener información de routing desde el propio Kernel. \\
\par
Por último, se quiere mencionar la diferencia que existe entre ambas tecnologías en la interfaz de control. En P4 la interfaz de control serán las tablas, las cuales se definirán su estructura desde el propio programa P4 pero se irán completando vía P4Runtime por un controlador externo. Por ejemplo, ONOS\footnote{\url{https://wiki.onosproject.org/display/ONOS/P4+brigade}}, el cual soporta la configuración vía P4Runtime. En cambio, en \gls{xdp} se podría decir que el equivalente a las tablas serían los mapas \gls{bpf}.\\
\par
Estos mapas, de tipo clave-valor, son definidos por el programa que se ancla en el Kernel, y generalmente son completados por programas de espacio de usuario, que acceden a éste a través de descriptores de archivo. Se podría decir que es un sistema al que aún le queda recorrido para ser igual de consistente que las tablas en P4, dado que sería necesario el desarrollo de un servidor \gls{xdp} de espacio de usuario que estuviera a la escucha de información de control. Este servidor se encargaría de escribir en los mapas \gls{bpf} la información equivalente de control que permitiese que el programa anclado en el Kernel actuara acorde a las directrices de un hipotético controlador.\\
\par

Por tanto, se concluye indicando que la tecnología \gls{xdp} actualmente está mejor enfocada para escenarios de integración totales (Ver figura \ref{sdn_iot_total}), donde el dispositivo \gls{iot} únicamente dispone de interfaces wireless. Este dispositivo se beneficiará del rendimiento y la optimización de recursos que ofrece \gls{xdp} al actuar de forma reactiva a los paquetes. \\
\par
En cuanto a la tecnología P4 se cree que está más enfocada a escenarios de integración parcial (Ver figura \ref{sdn_iot_parcial}), donde el dispositivo \gls{iot} mediador dispone tanto de interfaces cableadas para comunicarse con el core \gls{sdn}, como de una interfaz inalámbrica para comunicarse con otros dispositivos \gls{iot}. Dichos dispositivos se beneficiarán de las facilidades que otorga P4 para realizar multicast, y de la interfaz de control tan bien definida que tiene para ser controlado por un controlador \gls{sdn} convencional.
\vspace{2cm}




\newpage
%%%%%%%%%%%%%%%%%%%%%%%%%%%%%%%%%%%%%%%%%%%%%%%%%%%%%%%%%%%%%%%%%%%%%%%%%%%%%%%%%%%%%%%%%%%%%%%%%
\section{Líneas de trabajo futuro}
\label{trabajoFuturo}

Como se ha podido apreciar a lo largo del \gls{tfg}, la tecnología \gls{xdp} aun se está asentando; se está abriendo camino dando nuevas posibilidades a los desarrolladores de la parte de \textit{Networking} del Kernel, reescribiendo herramientas y distintos procesos del \textit{stack} de red del Kernel de Linux\footnote{\url{https://github.com/xdp-project/net-next}}. Por otro lado, se encuentra la tecnología P4, la cual ya lleva un par de años en escena, pero cuya comunidad va aumentando día tras día,  con ello, la calidad y robustez de sus implementaciones.   \\
\par
Desgraciadamente, aunque estas dos tecnologías estén en constante crecimiento, no parece que a día de hoy haya un interés especial por llevarlas al \gls{iot}. Debido a lo cual, a lo largo de este \gls{tfg} no se ha encontrado la información suficiente, ni las plataformas adecuadas para realizar el estudio y análisis de la integración de los dispositivos \gls{iot} en entornos \gls{sdn}, llegando incluso a tener que realizar implementaciones personalizadas de plataformas para la emulación de redes wireless con finalidad de poder evaluar los casos de uso. \\
\par
El hecho que de que aún no haya un interés general por llevar a dichas tecnologías al mundo \gls{iot}, deja mucho por hacer, explorar, desarrollar y probar. A continuación, se indican las vías de trabajo futuro que se cree que abrirán el camino a futuros trabajos e implementaciones, que son principalmente dos: la manipulación directa de las cabeceras \texttt{ieee80211} y la evaluación en redes de baja capacidad (\texttt{ieee802154}).

%%%%%%%%%%%%%%%%%%%%%%%%%%%%%%%%%%%%%%%%%%%%%%%%%%%%%%%%%%%%%%%%%%%%%%%%%%%%%%%%%%%%%%%%%%%%%%%%%
