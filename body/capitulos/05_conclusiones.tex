\chapter{Conclusiones y trabajo futuro}
\label{conclusiones}

En este capítulo, se concluirá la memoria presentando las conclusiones del trabajo realizado y evaluando tanto el trabajo en sí como los resultados obtenidos durante el desarrollo. Además, se explorarán las posibles vías de trabajo futuro que surgieron a medida que se llevaba a cabo el TFM. Se presentarán estas vías y se cerrará la memoria con una visión general y reflexiva sobre el proyecto realizado.


%%%%%%%%%%%%%%%%%%%%%%%%%%%%%%%%%%%%%%%%%%%%%%%%%%%%%%%%%%%%%%%%%%%%%%%%%%%%%%%%%%%%%%%%%%%%%%%%%
\section{Conclusiones del \glsentryshort{tfm}}

La finalidad de los casos de uso era la de ver con qué tecnología (\gls{xdp}, P4) es más sencillo y eficiente definir el \textit{datapath} de los dispositivos \gls{iot} en entornos \gls{sdn}. Por tanto, antes de realizar las conclusiones, se deberá revisar los resultados de las evaluaciones de ambas tecnologías en los distintos entornos, cableado e inalámbrico. \\
\par
Como se puede apreciar en la tabla \ref{tab:useCases}, con la tecnología \gls{xdp} en medios cableados no se pudo realizar un Broadcast de forma nativa, es decir, dicha tecnología no fue suficiente para lograr el broadcast. Es verdad que dicha limitación fue estudiada, y se planteó una solución, haciendo uso de un programa \gls{bpf} adicional en el \gls{tc}, pero aún teniendo en cuenta que se solucionó, no se puede afirmar que la tecnología lo soporte.  En cambio, si se observa dicha funcionalidad en un entorno Wireless, ya no habría dicha limitación dado que no sería necesario clonar los paquetes. Generalmente cambiando la dirección destino del paquete de capa dos a difusión, y transmitiéndolo al medio, será suficiente para lograr el Broadcast. \\
\par

En cambio, la tecnología P4, a diferencia de \gls{xdp}, tiene una interfaz de alto nivel para definir funcionalidades de broadcast/multicast. Esta interfaz, comúnmente conocida como ``grupos multicast", permite definir en una estructura de datos de tipo JSON por qué puertos inundar y con qué cantidad de réplicas por puerto. De forma adicional, cabe mencionar que los grupos multicast permiten modificar el tipo de instancia de los paquetes clonados, siendo posible diferenciar los paquetes originales de los clonados. \\
\par

\begin{table}[ht]
\centering
\begin{tabular}{|l|c|c|c|c|}
\hline
\rowcolor[HTML]{EFEFEF} 
\multicolumn{1}{|c|}{\cellcolor[HTML]{EFEFEF}\textbf{Caso de uso}} & \textbf{XDP} & \textbf{P4} & \textbf{P4-Wireless} & \textbf{XDP-Wireless} \\ \hline
case01 - Drop                                                      & \cmark       & \cmark      & \cmark               & \cmark                \\ \hline
case02 - Pass                                                      & \cmark       & \xmark      & \xmark               & \cmark                \\ \hline
case03 - Echo server                                               & \cmark       & \cmark      & \cmark               & \cmark                \\ \hline
case04 - Layer 3 forwarding                                        & \cmark       & \cmark      & \cmark               & \cmark                \\ \hline
case05 - Broadcast                                                 & \xmark       & \cmark      & \cmark               & \cmark                \\ \hline
\end{tabular}
\caption{Resumen sobre los casos de uso desarrollados}
\label{tab:useCases}
\end{table}

Volviendo a la tabla \ref{tab:useCases}, se quiere hacer especial hincapié en que la tecnología P4 no soporta la funcionalidad de dejar pasar o delegar los paquetes, sin afectarles el plano de datos programado en P4. Por otro lado, con \gls{xdp} siempre se puede pasar el paquete al Kernel para que se encargue él del procesamiento. Sin embargo, en P4, se debe definir de forma exclusiva todo el \textit{datapath}, por lo que no hay a quien delegar el paquete, es decir, se tiene que encargar el propio programa P4. Este punto es muy positivo para \gls{xdp}, ya que acciones sencillas y repetitivas pueden ser definidas en un programa \gls{xdp} y delegar el resto de funcionalidades al Kernel. Por tanto, se actuaría de forma cooperativa y ganando en rendimiento como se vio en el case04 (\ref{xdp_ether_case04}), donde se trabajaba de forma conjunta para obtener información de routing desde el propio Kernel. \\
\par
Por último, se quiere mencionar la diferencia que existe entre ambas tecnologías en la interfaz de control. En P4 la interfaz de control serán las tablas, las cuales se definirán su estructura desde el propio programa P4 pero se irán completando vía P4Runtime por un controlador externo. Por ejemplo, ONOS\footnote{\url{https://wiki.onosproject.org/display/ONOS/P4+brigade}}, el cual soporta la configuración vía P4Runtime. En cambio, en \gls{xdp} se podría decir que el equivalente a las tablas serían los mapas \gls{bpf}.\\
\par
Estos mapas, de tipo clave-valor, son definidos por el programa que se ancla en el Kernel, y generalmente son completados por programas de espacio de usuario, que acceden a éste a través de descriptores de archivo. Se podría decir que es un sistema al que aún le queda recorrido para ser igual de consistente que las tablas en P4, dado que sería necesario el desarrollo de un servidor \gls{xdp} de espacio de usuario que estuviera a la escucha de información de control. Este servidor se encargaría de escribir en los mapas \gls{bpf} la información equivalente de control que permitiese que el programa anclado en el Kernel actuara acorde a las directrices de un hipotético controlador.\\
\par

Por tanto, se concluye indicando que la tecnología \gls{xdp} actualmente está mejor enfocada para escenarios de integración totales (Ver figura \ref{sdn_iot_total}), donde el dispositivo \gls{iot} únicamente dispone de interfaces wireless. Este dispositivo se beneficiará del rendimiento y la optimización de recursos que ofrece \gls{xdp} al actuar de forma reactiva a los paquetes. \\
\par
En cuanto a la tecnología P4 se cree que está más enfocada a escenarios de integración parcial (Ver figura \ref{sdn_iot_parcial}), donde el dispositivo \gls{iot} mediador dispone tanto de interfaces cableadas para comunicarse con el core \gls{sdn}, como de una interfaz inalámbrica para comunicarse con otros dispositivos \gls{iot}. Dichos dispositivos se beneficiarán de las facilidades que otorga P4 para realizar multicast, y de la interfaz de control tan bien definida que tiene para ser controlado por un controlador \gls{sdn} convencional.
\vspace{2cm}




\newpage
%%%%%%%%%%%%%%%%%%%%%%%%%%%%%%%%%%%%%%%%%%%%%%%%%%%%%%%%%%%%%%%%%%%%%%%%%%%%%%%%%%%%%%%%%%%%%%%%%
\section{Líneas de trabajo futuro}
\label{trabajoFuturo}

Tras concluir y revisar el alcance conseguido de los objetivos propuestos para la ejecución del proyecto, podemos apuntar a próximas líneas de trabajo futuro basadas en los siguientes puntos.
\begin{itemize}
    \item Realizar un despliegue en real utilizando Raspberry Pi (RPi) que tienen una interfaz WiFi nativa para evaluar el funcionamiento del in-BOFUSS con varias RPis. Esto permitirá verificar si el rendimiento es similar al observado durante el proyecto.

    \item Realizar la integración del switch BOFUSS en la jerarquía de clases de Mininet-WiFi. Esto facilitará la realización de pruebas y evaluaciones más consistentes al combinar las funcionalidades de ambos proyectos.

    \item Añadir seguridad mediante el uso de TLS en las conexiones OpenFlow in-band desde el UserAP hacia el controlador. Esto garantizará la confidencialidad y la integridad de las comunicaciones, protegiendo la información transmitida entre el UserAP y el controlador.

    \item Realizar simulaciones de pérdida utilizando el protocolo desarrollado y evaluar cómo afecta al funcionamiento intrínseco de la operativa programada. Esto ayudará a comprender el impacto de la pérdida de paquetes en el rendimiento del sistema y permitirá optimizar el protocolo en función de los resultados obtenidos.

    \item Explorar la posibilidad de tener dos UserAPs en el espacio de nombres raíz (root namespace). Actualmente, solo se permite tener uno para evitar el bypass de paquetes. Investigar cómo gestionar la presencia de dos UserAPs y evaluar cómo afecta a la conectividad y al rendimiento del sistema.
\end{itemize}

Estas líneas de trabajo futuro proporcionan oportunidades para mejorar y ampliar el proyecto, explorando nuevas funcionalidades, mejorando la seguridad y evaluando su rendimiento en diferentes escenarios y condiciones. Como se ha indicado en las conclusiones, aparte de apuntar a las posibles líneas de trabajo a futuro, también hay que mencionar la realidad inmediata, que después de estar trabajando con el \textit{software switch} BOFUSS, hacerle funcionar en las últimas versiones se tomará el relevo a Eder como maintainer del proyecto en una nueva organización agnóstica de organizaciones.


%%%%%%%%%%%%%%%%%%%%%%%%%%%%%%%%%%%%%%%%%%%%%%%%%%%%%%%%%%%%%%%%%%%%%%%%%%%%%%%%%%%%%%%%%%%%%%%%%
