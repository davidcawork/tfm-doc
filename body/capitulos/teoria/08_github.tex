\section{Contribuciones en GitHub}
\label{sec:estadoArte_github}


El proyecto GitHub es una plataforma que proporciona alojamiento de repositorios. En el presente Trabajo de Fin de Máster (TFM), se utilizará la herramienta de control de versiones Git\footnote{\url{https://git-scm.com/}}, junto con GitHub como plataforma para alojar el código desarrollado. Sin embargo, GitHub no se utilizará únicamente para almacenar el código del TFM, sino que se aprovechará su carácter público para ofrecer documentación y ejemplos a todos los usuarios interesados que visiten el repositorio. \\
\par
\begin{itemize}
    \item Enlace al repositorio del \gls{tfm}: \url{https://github.com/davidcawork/TFM}
    \item Enlace al repositorio de la tesis: \url{https://github.com/davidcawork/tfm-thesis}
\end{itemize}
\vspace{0.5cm}

El repositorio incluye ficheros \texttt{README.md} en todos los directorios, que proporcionan explicaciones y análisis teóricos que permiten al visitante comprender la naturaleza de las pruebas, sus objetivos y las conclusiones que se pueden extraer de ellas. El propósito del repositorio es doble, ya que no solo almacena el código, sino que también ayuda a difundir los contenidos del proyecto. Además, se han ofrecido contribuciones útiles que pueden beneficiar a otros repositorios a través de solicitudes de extracción (pull requests). Durante este proyecto se ha contribuido de forma muy activa en varios proyectos opensource dado que la naturaleza del mismo abarcaba varias herramientas.  La tabla \ref{tab:githubContr} muestra todas las contribuciones realizadas.\\
\\

\begin{table}[ht]
    \centering
    \resizebox{\textwidth}{!}{%
        \begin{tabular}{|l|l|}
            \hline
            \rowcolor[HTML]{EFEFEF}
            \multicolumn{1}{|c|}{\cellcolor[HTML]{EFEFEF}{\color[HTML]{24292E} \textbf{Contribución}}} & \multicolumn{1}{c|}{\cellcolor[HTML]{EFEFEF}{\color[HTML]{24292E} \textbf{Enlace al Pull-Request}}} \\ \hline
            Dar soporte al bmv2 entorno P4 en Mininet-WiFi                                             & \url{https://github.com/intrig-unicamp/mininet-wifi/pull/302}                                       \\ \hline
            Arreglar la compilación del BOFUSS en Mininet-WiFi                                         & \url{https://github.com/intrig-unicamp/mininet-wifi/pull/495}                                       \\ \hline
            Aclarar el uso de ONOS-gui                                                                 & \url{https://groups.google.com/a/opennetworking.org/g/onos-dev/c/mbfyCaoeCdU}                       \\ \hline
            Resolver problema interfaz dpctl con el BOFUSS                                             & \url{https://github.com/mininet/mininet/issues/745}                                                 \\ \hline
            Dar escenarios funcionales en la 22.04 de Mininet con Onos                                 & \url{https://groups.google.com/a/opennetworking.org/g/onos-dev/c/a0VjA0Xw_-M}                       \\ \hline
        \end{tabular}%
    }
    \caption{Resumen de contribuciones realizadas durante todo el proyecto del \glsentryshort{tfm}}
    \label{tab:githubContr}
\end{table}