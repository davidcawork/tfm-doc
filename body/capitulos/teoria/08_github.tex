\section{Contribuciones en GitHub}
\label{estadoArte_github}


La herramienta GitHub \cite{github2016github} es una plataforma para alojar repositorios de forma remota. En este \gls{tfg}, se hará uso de la herramienta de control de versiones Git\footnote{\url{https://git-scm.com/}}, y de GitHub como plataforma para alojar el código. Pero no se hará un uso exclusivo de la plataforma para almacenar el código desarrollado en el \gls{tfg}, sino que se aprovechará el carácter público del repositorio para ofrecer documentación y ejemplos a todos los usuarios interesados que lo visiten. \\
\par
\begin{itemize}
    \item Enlace al repositorio del \gls{tfg}: \url{https://github.com/davidcawork/TFG} 
\end{itemize}
\vspace{0.5cm}
Todo el proceso de documentación en el repositorio pasa por los ficheros \texttt{README}, los cuales se podrán encontrarán en todos los directorios del repositorio. Estos ficheros suministrarán la información necesaria a los visitantes para poder replicar las pruebas realizadas, y hacer uso del \textit{software} desarrollado. Pero además, se han añadido las explicaciones y  los análisis teóricos necesarios, para que el visitante realmente entienda la naturaleza de las pruebas, que se espera de ellas y que conclusiones se podrán sacar de dichos test.  \\
\par

La finalidad del repositorio por tanto es doble, ya que servirá para almacenar el código, pero también ayudará a divulgar los contenidos de este proyecto. De forma adicional, todos los desarrollos útiles y que pueden aportar en otros repositorios se han ofrecido en forma de contribución vía \textit{pull-request}. A continuación, se en la tabla \ref{tab:githubContr} se indican  todas las contribuciones realizadas.\\
\par

\begin{table}[ht]
\centering
\resizebox{\textwidth}{!}{%
\begin{tabular}{|l|l|}
\hline
\rowcolor[HTML]{EFEFEF} 
\multicolumn{1}{|c|}{\cellcolor[HTML]{EFEFEF}{\color[HTML]{24292E} \textbf{Contribución}}} & \multicolumn{1}{c|}{\cellcolor[HTML]{EFEFEF}{\color[HTML]{24292E} \textbf{Enlace al Pull-Request}}} \\ \hline
Nuevo método de instalación de todas las dependencias del entorno P4                       & \url{https://github.com/p4lang/tutorials/pull/261}                                                        \\ \hline
Corregir documentación del repositorio XDP Tutorial                                        & \url{https://github.com/xdp-project/xdp-tutorial/pull/95}                                               \\ \hline
Integración del BMV2 en Mininet-Wifi                                                       & \url{https://github.com/intrig-unicamp/mininet-wifi/pull/302}                                             \\ \hline
Arreglar interfaz gráfica cuando los APs tienen movilidad                                  & \url{https://github.com/intrig-unicamp/mininet-wifi/pull/229}                                             \\ \hline
Dar soporte de las estaciones Wifi en el comando pingallfull                               & \url{https://github.com/intrig-unicamp/mininet-wifi/pull/230}                                             \\ \hline
\end{tabular}%
}
\caption{Resumen de contribuciones realizadas}
\label{tab:githubContr}
\end{table}