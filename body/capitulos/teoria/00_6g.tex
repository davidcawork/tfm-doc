%%%%%%%%%%%%%%%%%%%%%%%%%%%%%%%%%%%%%%%%%%%%%%%%%%%%%%%%%%%%%%%%%%
\section{Red de comunicación \glsentryshort{6g}}
\label{sec:6g}

Las redes de comunicación de sexta generación, están en desarrollo para ofrecer una conectividad aún más rápida, confiable y eficiente que las generaciones anteriores. A medida que la demanda de datos continúa creciendo exponencialmente y surgen nuevas aplicaciones y tecnologías, como el \gls{iot}, la realidad aumentada y la \gls{ai}, se espera que el \gls{6g} proporcione una infraestructura de red sólida y escalable para satisfacer estas necesidades futuras. Los principales puntos claves de esta nueva generación se pueden resumir en los siguientes puntos.

\begin{itemize}

\item Velocidad y capacidad extremadamente altas: Una de las principales características del \gls{6g} será la velocidad y capacidad extremadamente altas, superando con creces las capacidades del \gls{5g}. Se espera que el \gls{6g} alcance velocidades de terabits por segundo, lo que permitirá una transmisión de datos ultra rápida y un soporte eficiente para aplicaciones de alta demanda, como son la transmisión de video en 8K, realidad virtual y realidad aumentada de alta calidad.

\item Latencia ultrabaja: El \gls{6g} aspira a lograr una latencia ultrabaja, reduciendo aún más el tiempo de respuesta de la red. Se espera que la latencia en el \gls{6g} sea de solo unos pocos milisegundos, lo que permitirá aplicaciones en tiempo real de misión crítica, como cirugías remotas, vehículos autónomos y aplicaciones de realidad virtual y aumentada altamente interactivas.

\item Conectividad ubicua: El \gls{6g} tiene como objetivo proporcionar conectividad ubicua, extendiéndose más allá de las áreas urbanas y llegando a áreas remotas y rurales haciendo uso de radio enlaces satelitales de orbita baja (LEO). Se espera que el \gls{6g} aborde la brecha digital y brinde conectividad global a nivel mundial, permitiendo una mayor inclusión digital y oportunidades equitativas para todos.

\item Integración de tecnologías emergentes: El \gls{6g} se construirá sobre tecnologías emergentes, como inteligencia artificial (IA), \gls{ml}, llegando a proponer la computación cuántica y nanotecnología. Estas tecnologías avanzadas permitirán el desarrollo de sistemas de red más inteligentes y autónomos, optimizando la eficiencia y la capacidad de adaptación de la red.

\item Sostenibilidad y eficiencia energética: El \gls{6g} se enfocará en la sostenibilidad y la eficiencia energética para reducir su impacto ambiental. Se espera que las redes \gls{6g} sean más eficientes en términos de consumo de energía, al tiempo que brinden una mayor capacidad y rendimiento. Además, se explorarán nuevas técnicas de transmisión de energía y comunicación inalámbrica para impulsar la eficiencia energética en dispositivos y redes.

\end{itemize}

\subsection{Tecnologías habilitadoras}

Como se comentó en el capítulo de introducción (Capitulo \ref{ch:intro}), esta nueva generación aún es prematura y no tiene unos estándares claros y definidos sobre como se va a llevar a cabo, sin embargo, ya empieza a haber propuestas sobre las tecnologías habilitadoras que harán del \gls{6g} una realidad tarde o temprano. Desde el proyecto líder europeo  6G-Flagship y la organiazción One6G ya se apuntan a una serie de tecnologías claves, las cuales, se indican a continuación.

\subsubsection{Banda de los THz}

Según se ha indicado, con el \gls{6g}, se espera que fusione los mundos digital y físico en todas sus dimensiones, brindando a los usuarios una experiencia holográfica, háptica y multisensorial. Según la Unión Internacional de Telecomunicaciones (UIT), estas aplicaciones surgirán en la próxima década y se caracterizarán por requerir comunicaciones altamente exigentes [1]. Además, algunas aplicaciones demandarán funcionalidades que los sistemas celulares actuales no proporcionan, como una detección precisa, mapeo y localización. Un ejemplo destacado de caso de uso es la telepresencia holográfica. La transmisión de hologramas 3D en su forma más básica requiere una capacidad de más de 4 Tbps [2]. La capacidad de detectar y comprender el entorno permitirá a la red predecir el movimiento de los usuarios sin información explícita, creando así una experiencia inmersiva a distancia. Asimismo, las comunicaciones de alta velocidad de datos y baja latencia necesarias para la automatización de fábricas se beneficiarán de las transmisiones a terabits por segundo (Tbps). \\
\\
Con el fin de satisfacer esta necesidad, las comunicaciones en terahercios (THz) se han identificado como una opción prometedora para la capa física en 6G, ya que tienen el potencial de permitir velocidades de datos de terabits por segundo (Tbps) y ofrecer servicios de detección, mapeo y localización [3]. En la Conferencia Mundial de Radiocomunicaciones de 2019 (CMR2019), la Unión Internacional de Telecomunicaciones (UIT) identificó un espectro de 137 GHz entre 275 y 450 GHz que se puede utilizar para las comunicaciones en terahercios [4]. Esto se suma al espectro ya asignado por debajo de los 275 GHz, lo que proporciona un total de 160 GHz en la gama de sub-terahercios. En 2017, el grupo de trabajo IEEE 802 completó el primer estándar inalámbrico para frecuencias portadoras alrededor de los 300 GHz (IEEE Std 802.15.3d-2017) [5][39]. Esto establece una base sólida para el desarrollo y la implementación de las comunicaciones en terahercios en el contexto de 6G.

\subsubsection{Próxima generación de \glsentryshort{mimo}}

\subsubsection{\glsentryshort{ai} federada y distribuida}
 
\subsubsection{Plano de datos inteligente}

\subsubsection{Infraestructuras flexibles y programables }
   



\subsection{Casos de uso verticales}



En resumen, el \gls{6g} promete llevar la conectividad y las comunicaciones inalámbricas a un nuevo nivel, superando las capacidades del \gls{5g} y habilitando una amplia gama de aplicaciones y servicios avanzados. Con velocidades ultra altas, latencia ultrabaja, conectividad ubicua y la integración de tecnologías emergentes, el \gls{6g} está destinado a impulsar la transformación digital en diversas industrias y habilitar un futuro aún más conectado e inteligente.

