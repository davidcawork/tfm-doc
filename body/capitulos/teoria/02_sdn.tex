\section{Redes  \glsentryshort{sdn}}
\label{sec:sdn}


El paradigma \gls{sdn} \cite{nadeau2013sdn} se refiere a una arquitectura de red en la que se separa el plano de control de la red para centralizarlo en un controlador único. Esta estructura permite lograr una administración de red más centralizada y flexible \cite{nadeau2013sdn}. La idea del \gls{sdn} comenzó a gestarse en la Universidad de Stanford en 2003, cuando el profesor asociado de ese entonces, Nick McKeown, planteó las limitaciones de las redes convencionales y la necesidad de replantear cómo operaban los \textit{backbones} \cite{sdnBegins}. En 2011, se acuñó el término \gls{sdn}, al mismo tiempo que se lanzó la organización \gls{onf} \cite{onf}, encargada de establecer estándares y promover la difusión del \gls{sdn}.

\subsection{Arquitectura del paradigma \glsentryshort{sdn}}

La arquitectura \gls{sdn} se destaca por su dinamismo, rentabilidad y adaptabilidad, lo que la convierte en una solución ideal para las demandas actuales de las redes de comunicaciones. Como se ha mencionado anteriormente, esta arquitectura se basa en la separación del plano de control y el plano de datos, trasladando el control a una entidad central llamada controlador. A través de esta entidad, se ofrecen interfaces que permiten a las aplicaciones de servicios de red utilizarlas. Esto hace que el control de la red sea programable directamente, lo que agiliza y dinamiza su gestión.\\
\\
La arquitectura \gls{sdn} se divide en tres capas, como se muestra en la figura \ref{fig:sdnBasicArch}. La primera capa es el plano de datos, donde se encuentran todos los elementos de red responsables del reenvío de los datos. La segunda capa es el plano de control, compuesto por diferentes controladores \gls{sdn}. Por último, la capa de aplicación alberga todas las aplicaciones que se comunican con el controlador \gls{sdn}.\\
\\
Estas capas se comunican entre sí a través de interfaces abiertas. Por ejemplo, la interfaz \textit{Southbound} permite programar el estado de reenvío de los elementos de red en el plano de datos. Por otro lado, la interfaz \textit{Northbound} permite la comunicación entre las aplicaciones y los controladores \gls{sdn}, permitiendo la obtención de datos y el ajuste de parámetros mediante una API-Rest. Además, existen otras interfaces, como \textit{Westbound} y \textit{Eastbound}, que se han consolidado en los últimos años para interconectar controladores y establecer políticas comunes entre diferentes dominios \gls{sdn}.\\

% Foto 
\begin{figure}[ht]
    \centering
    \includegraphics[width=0.9\textwidth]{archivos/img/teoria/sdn_arch.png}
    \caption{Arquitectura típica \glsentryshort{sdn} \cite{carrascal2020diseno}}
    \label{fig:sdnBasicArch}
\end{figure}

\subsection{OpenFlow}

Existen varios protocolos para el control de los elementos de red desde el controlador, pero el más utilizado es Openflow. OpenFlow es un protocolo de la interfaz \textit{Southbound} que comunica los controladores \gls{sdn} con los elementos de red para configurar el estado de reenvío de estos últimos. La especificación de este protocolo se encuentra recogida por la \gls{onf}\footnote{\url{https://www.opennetworking.org/software-defined-standards/specifications/}}, contando con numerosas versiones siendo la última la versión \texttt{1.5.1} del 2015. \\
\par

El elemento clave de OpenFlow es el flujo (\textit{flow}), los cuales se conforman de paquetes que ha sido clasificados en función de reglas. Dichas reglas se encuentran en las tablas de flujo (\textit{flow table}) y  suelen estar relacionadas con los puertos de entrada o valores de cabecera típicos. Cuando estos criterios coinciden con los del paquete entrante se produce un \textbf{\textit{match}}.\\
\par
En el momento en que se produce un \textit{match}, el paquete en cuestión se verá sujeto a una serie de instrucciones asociadas a la regla con la que a hecho \textit{matching}. Estas instrucciones pueden ir desde, hacer una medición del paquete, aplicar una acción ó ir a otra tabla de flujo. De esta forma, con unas tablas de flujo completadas con unas reglas suministradas por el controlador \gls{sdn}, se conforma el estado de reenvío del switch en cuestión \cite{nadeau2013sdn}. 
