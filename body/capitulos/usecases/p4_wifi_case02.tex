\subsection{Case02 - Pass}
\label{p4_wifi_case02}


Como se puede apreciar, en este directorio no hay ningún programa P4, al igual que en caso de uso P4, case02 (\ref{P4_ether_case02}),  en entornos cableados. Esto se debe a que no hay equivalente en P4 del código de retorno \texttt{XDP\_PASS}, por ello, no se puede hacer nada en este caso de uso. El código de de retorno en \gls{xdp} es una forma para llevar a cabo una acción con el paquete que llega a la interfaz, en la cual hay anclado un programa \gls{xdp}. En este caso el código de retorno  \texttt{XDP\_PASS} implica que el paquete se pasa al siguiente punto del procesamiento del \textit{stack} de red en el Kernel de Linux. Es decir, si el programa está anclado a la \gls{nic}, se dejará pasar el paquete al \gls{tc}, de ahí al propio \textit{stack} de red para parsear sus cabeceras, y más adelante, dárselo de comer a la interfaz de sockets.\\
\par

En P4, el entorno donde se llevaran a cabo los casos de uso será Mininet-WiFi con \gls{ap}s (\gls{bmv2}) y host. Los \gls{ap}s (\gls{bmv2}) son un software switch en los que podemos inyectar código P4, con el cual podemos definir el \textit{datapath} del mismo.\\
\par

Ahora bien, aquí viene la gran diferencia entre ambas tecnologías, con \gls{xdp} siempre es posible pasarle el paquete al Kernel para que se encargue el del procesamiento, pero en P4, debemos definir nosotros de forma exclusiva todo el \textit{datapath}, por lo que no hay a quien delegar el paquete, se debe encargar el propia programa P4. Entonces, como tal, no habría equivalente del  \texttt{XDP\_PASS} en P4, como es obvio ni en escenarios cableados ni inalámbricos, es una característica de la propia tecnología.\\
\par