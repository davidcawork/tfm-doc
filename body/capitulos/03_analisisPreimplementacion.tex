\chapter{Diseño del protocolo de control In-Band}
\label{ch:analisis}

En este capítulo, se abordará una fase fundamental del proyecto, centrada en el diseño de un protocolo de control In-Band para la gestión de redes. En este capítulo, se realizará un exhaustivo análisis de soluciones anteriores basadas en el enfoque In-Band, donde se explorarán diferentes propuestas y se evaluarán sus fortalezas y debilidades.\\
\\
El objetivo principal será definir las funcionalidades básicas que debe poseer el protocolo de control In-Band, considerando los requisitos específicos del proyecto y las necesidades de los entornos de redes actuales. Se examinarán aspectos clave, como la capacidad de establecer una conexión entre los nodos de la red y el controlador, el manejo eficiente del plano de datos para la transmisión de información de control y la escalabilidad para adaptarse a entornos de redes heterogéneas y de gran tamaño. Además, se proporcionará una explicación detallada del funcionamiento del protocolo diseñado, describiendo los diferentes componentes, los mensajes intercambiados entre nodos y controlador, así como los procedimientos de configuración y gestión de la red. Se analizarán las decisiones de diseño tomadas y se justificarán en base a los objetivos del proyecto y las características de los entornos de redes abordados.Por último, se tomará una decisión sobre la plataforma más adecuada para la implementación del protocolo de control In-Band. Se evaluarán diferentes opciones, considerando factores como la disponibilidad de herramientas y tecnologías relevantes, la compatibilidad con los requisitos del proyecto y la viabilidad de su implementación en entornos reales.

%%%%%%%%%%%%%%%%%%%%%%%%%%%%%%%%%%%%%%%%%%%%%%%%%%%%%%%%%%%%%%%%%%%%%%%%%%%%%%%%%%%%%%%%%%%%%%%%%
\section{Protocolo In-Band}
\label{sec:ana_inband}

%%%%%%%%%%%%%%%%%%%%%%%%%%%%%%%%%%%%%%%%%%%%%%%%%%%%%%%%%%%%%%%%%%%%%%%%%%%%%%%%%%%%%%%%%%%%%%%%%
\section{Plataforma de desarrollo y validación}
\label{sec:ana_mininet_wifi}

%%%%%%%%%%%%%%%%%%%%%%%%%%%%%%%%%%%%%%%%%%%%%%%%%%%%%%%%%%%%%%%%%%%%%%%%%%%%%%%%%%%%%%%%%%%%%%%%%
\section{Agente \glsentryshort{sdn}}
\label{sec:ana_switch}

%%%%%%%%%%%%%%%%%%%%%%%%%%%%%%%%%%%%%%%%%%%%%%%%%%%%%%%%%%%%%%%%%%%%%%%%%%%%%%%%%%%%%%%%%%%%%%%%%
\section{Agente de control \glsentryshort{sdn}}
\label{sec:ana_controller}

%%%%%%%%%%%%%%%%%%%%%%%%%%%%%%%%%%%%%%%%%%%%%%%%%%%%%%%%%%%%%%%%%%%%%%%%%%%%%%%%%%%%%%%%%%%%%%%%%
\section{Análisis funcional de la interfaz del \glsentryshort{bofus}}
\label{sec:ana_bofuss}

%%%%%%%%%%%%%%%%%%%%%%%%%%%%%%%%%%%%%%%%%%%%%%%%%%%%%%%%%%%%%%%%%%%%%%%%%%%%%%%%%%%%%%%%%%%%%%%%%
\section{Análisis de la clase \texttt{UserAP} en Mininet-WiFi}
\label{sec:ana_userap}

%%%%%%%%%%%%%%%%%%%%%%%%%%%%%%%%%%%%%%%%%%%%%%%%%%%%%%%%%%%%%%%%%%%%%%%%%%%%%%%%%%%%%%%%%%%%%%%%%
\section{Análisis del entorno de depuración del \glsentryshort{bofus}}
\label{sec:ana_gdb}


%%%%%%%%%%%%%%%%%%%%%%%%%%%%%%%%%%%%%%%%%%%%%%%%%%%%%%%%%%%%%%%%%%%%%%%%%%%%%%%%%%%%%%%%%%%%%%%%%
