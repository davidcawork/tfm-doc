\section{Análisis funcional de la interfaz del \glsentryshort{bofus}}
\label{sec:ana_bofuss}

En esta sección, exploraremos en detalle el análisis funcional de la interfaz del \gls{bofus}, centrándonos específicamente en los binarios que componen su arquitectura. Estos binarios, conocidos como ofdatapath y ofprotocol, desempeñan un papel fundamental en la operativa básica de este switch de espacio de usuario.\\
\\
El ofdatapath, como su nombre sugiere, es responsable de procesar el plano de datos en el \gls{bofus}. Este componente se encarga de recibir, analizar y tomar decisiones en función de los paquetes que circulan por la red. A través de técnicas como la clasificación, el enrutamiento y la manipulación de paquetes, el ofdatapath garantiza una transferencia de datos fluida y eficiente en el entorno OpenFlow. Por otro lado, el ofprotocol se ocupa del agente de control en el \gls{bofus}. Su función principal consiste en establecer la comunicación entre el controlador y el switch. A través del ofprotocol, el controlador puede enviar instrucciones y recibir información sobre el estado de la red. Esto permite una gestión centralizada y dinámica de las políticas de red, facilitando la adaptación y optimización de la infraestructura según las necesidades del entorno.\\


\subsection{Binario \texttt{ofprotocol}}

\subsection{Binario \texttt{ofdatapath}}