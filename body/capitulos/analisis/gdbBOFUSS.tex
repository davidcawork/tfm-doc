\section{Análisis del entorno de depuración del \glsentryshort{bofus}}
\label{sec:ana_gdb}

En esta sección, exploraremos el proceso de depuración del \gls{bofus} utilizando Visual Studio Code (VS Code) y los conocimientos adquiridos sobre el funcionamiento de la interfaz de línea de comandos  del \gls{bofus} en Mininet-WiFi (Ver sección \ref{sec:ana_bofuss}). El objetivo es comprender en detalle cómo se ejecutan los comandos y qué sucede internamente durante la operación del \gls{bofus} en un entorno de red inalámbrica emulada. En nuestro escenario, trabajaremos con Mininet-WiFi, que nos proporciona un entorno virtualizado para la emulación de redes inalámbricas gracias al modulo del Kernel mac80211\_hwsim. Por ello, trabajaremos en estrecha colaboración con Mininet-WiFi para llevar a cabo la depuración. Sin embargo, este enfoque puede presentar cierta complejidad, por lo que realizaremos una primera aproximación ejecutando el código de una topología sencilla en modo de depuración, lo que nos permitirá observar los comandos que se ejecutan y comprender su funcionamiento. Posteriormente, exploraremos cómo convertir estos comandos en scripts de shell para mayor conveniencia y automatización. Nuestras herramientas de trabajo serán las siguientes:\\

\begin{itemize}
    \item Visual Studio Code (VS Code): Utilizaremos este editor para escribir y editar el código, así como para realizar la depuración paso a paso. VS Code proporciona una interfaz intuitiva y funciones avanzadas de depuración que nos facilitarán el proceso. A parte de tener una maravillosa terminal integrada y una interfaz con GDB ya desarrollada.

    \item Mininet-WiFi: Esta herramienta nos permitirá emular redes inalámbricas. Trabajaremos con una topología específica, la cual ya hemos mencionado en la sección anterior, y la ejecutaremos en modo de depuración para comprender mejor su funcionamiento, para así poder extraer las lineas de comandos necesarias para replicar su funcionamiento de forma completamente externa.

    \item GDB: Utilizaremos el depurador GDB para analizar y depurar el código del BOFUSS. GDB nos permitirá examinar el estado del programa en tiempo de ejecución, establecer puntos de interrupción, inspeccionar variables y ejecutar el código paso a paso, lo que nos ayudará a identificar posibles errores y problemas en el BOFUSS.
\end{itemize}

