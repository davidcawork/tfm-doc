\section{Plataforma de desarrollo y validación}
\label{sec:ana_mininet_wifi}

En este punto se va a valorar que plataforma se va a utilizar para desarrollar y validar el protocolo de comunicación in-band. Lo primera cuestión que se tiene que tener en cuenta cuando se va a elegir una plataforma de este tipo es, ¿Vamos a simular o a emular?. Aunque mucha gente considera que estos términos son equivalentes, no son lo mismo. Cuando hablamos de simulación nos referimos a un software que calcula el resultado de un evento dado un comportamiento esperado, donde la escala de tiempos de la simulación no se corresponde con la realidad. Es decir, el proceso a simular puede tener una duración de dos horas, pero como lo que se gestionan son los eventos asociados al proceso, el tiempo de la simulación solo vendrá regido por lo que tarde en procesar dichos eventos. Por otro lado, la emulación recrea el escenario bajo estudio en su totalidad en un hardware específico para luego estudiar su comportamiento. Donde la escala de tiempos de la emulación es el mismo que el tiempo real del proceso a emular. Un ejemplo para diferenciar ambos podría ser pensar en la cabina de un avión. Si jugáramos a un videojuego como Flight Simulator\footnote{\url{https://www.flightsimulator.com/}} estaríamos simulando el vuelo, pero si practicáramos utilizando una escala 1:1 (Ver figura \ref{fig:emu_simu}) con controles reales estaríamos hablando entonces de emulación.\\

%fig 1
\begin{figure}[ht!]
    \centering
    \includegraphics[width=0.8\textwidth]{archivos/img/analisis/emu_simu.jpeg}
    \caption{Entorno de emulación real de una cabina de avión a escala 1:1 \cite{emusim}}
    \label{fig:emu_simu}
\end{figure}

Una vez que ya tenemos claro cuales son las diferencias sobre la emulación y la simulación, vamos a ver que opciones tenemos en ambos ambitos para elegir una plataforma que se adecue a las necesidades del proyecto. De entre todas las opciones barajadas nos hemos quedados con las dos opciones en las cuales el autor tiene más experiencia, y en las cuales se pueden virtualizar tanto enlaces inalambricos como entornos \gls{sdn}.

\begin{itemize}
    \item Contiki-ng haciendo uso de su simulador integrado llamado Cooja, explicado anteriormente en la sección \ref{sec:contikiNG} del estado del arte.

    \item Mininet-WiFi, explicado anteriormente en la sección \ref{subsec:mininetWIFIS} del estado del arte.
\end{itemize}

La primera opción según se ha explicado ya es un entorno de simulación para dispositivos \gls{iot}, donde se podría llegar a meter un agente \gls{sdn} según se ha podido investigar \cite{baddeley2018evolving, Anadiotis:2019}. Esta opción es muy interesante dado que se utilizaría la tecnología radio más acorde a los dispositivos \gls{iot}, ya que implementan \texttt{ieee802154}. Sin embargo, lo malo de esta opción es que si bien es cierto que hay agentes \gls{sdn} ya disponibles, no hay agentes de control \gls{sdn} programados para Cooja, por lo que habría que programarse un controlador \gls{sdn} haciendo uso de la plataforma base de Contiki-ng para poder llegar a simular el protocolo. Por tanto el objetivo final del proyecto pasaría a un segundo plano, ya que programar un controlador \gls{sdn} desde cero no es una tarea sencilla.\\
\\
Por otro lado, tenemos a Mininet-WiFi o a su variante tambien \gls{iot}. Trabajar con una variante u otra es cuestión de indicar que módulo se carga en el Kernel de Linux, si \texttt{mac80211\_hwsim} o \texttt{mac802154\_hwsim}. En esta opción estaríamos emulando según se ha explicado anteriormente en el estado del arte. Recordemos que Mininet y todas las herramientas que nacen de ella hacen uso de las bondades del Kernel para ofrecer mecanismos de aislamiento para llegar a emular. Por ello, tendremos a nuestra disposición un entorno Linux donde a la par que emulamos podemos tener cualquier otra herramienta corriendo en la misma máquina. Esto es interesante por que abre la puerta a tener cualquier controlador \gls{sdn} corriendo en la máquina, teniendo tanto entorno inalámbrico como entorno \gls{sdn} ya habilitados, teniendo que preocuparnos exclusivamente en el desarrollo del protocolo.\\
\\
Ambas opciones presentan ventajas y desventajas, sin embargo, se ha tomado la decisión de seleccionar el entorno de emulación de Mininet-WiFi como plataforma para el desarrollo y validación. Esta elección se basa no solo en los aspectos mencionados anteriormente, sino también en el hecho de que todo el software desarrollado en un entorno Linux será fácilmente transferible a otros entornos Linux con hardware real. Por ejemplo, si se desea utilizar una Raspberry Pi, a pesar de que esta tiene una arquitectura ARM, se puede adaptar el bytecode generado para que sea compatible con dicha arquitectura, ya que el resto del software será compatible, haciendo que la implementación en nuevos targets sea casi inmediata.