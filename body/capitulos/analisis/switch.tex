\section{Agente \glsentryshort{sdn}}
\label{sec:ana_switch}

En este punto se tiene que valorar qué \textit{software switch} \gls{sdn} se va a utilizar. Se tendrán en cuenta las condiciones del entorno sobre el cual se van a desplegar los nodos \gls{sdn}, así como las características propias de cada switch, así como la facilidad y flexibilidad que nos entregue cada uno para desplegarlo sobre la plataforma emulada. Las opciones que se han considerado para este cometido son las siguientes:

\begin{itemize}
    \item \gls{ovs}, explicado anteriormente en la Sección \ref{subsec:OVS} del estado del arte.

    \item \gls{bofus}, explicado anteriormente en la Sección \ref{subsec:BOFUSS} del estado del arte.
\end{itemize}

Dado que cada herramienta tiene sus propias fortalezas y debilidades, es importante realizar una comparativa para determinar cuál de las dos opciones es más adecuada para nuestro caso de uso específico.

\begin{itemize}
    \item El \gls{ovs} tiene mucho más rendimiento que el \gls{bofus} dado que la mitad de su switch trabaja en espacio de Kernel, mientras que el \gls{bofus} como su nombre indica, trabaja en espacio de usuario.
    \item El \gls{ovs} es mucho más sólido, y por ende popular, dado que tiene una gran comunidad de desarrolladores que someten al software switch a una amplia variedad de test. Mientras que el \gls{bofus} no tiene un sistema de test, y hasta la fecha únicamente contaba con un mainteiner.
    \item El \gls{ovs} puede trabajar en modo standalone, o en modo software switch \gls{sdn}, mientras que el \gls{bofus} requiere de un agente de control que le rellene las tablas de flujo OpenFlow.
    \item El \gls{ovs} al trabajar en modo standalone se puede encontrar en la mayoría de entornos virtualizados, como infraestructura de red para interconectar instancias virtuales.
    \item Sin embargo, el \gls{ovs} al igual que tiene puntos a favor por trabajar a nivel de Kernel, también supone puntos negativos, como por ejemplo la complejidad que tiene para depurarlo y añadir nuevas funcionalidades en él. Mientras tanto, el \gls{bofus} al trabajar en espacio de usuario, permite una sencilla  depuración por ejemplo GDB, y es relativamente accesible añadirle nuevas funcionalidades.
\end{itemize}

% Please add the following required packages to your document preamble:
% \usepackage{graphicx}
% \usepackage[table,xcdraw]{xcolor}
% If you use beamer only pass "xcolor=table" option, i.e. \documentclass[xcolor=table]{beamer}
% Please add the following required packages to your document preamble:
% \usepackage{graphicx}
% \usepackage[table,xcdraw]{xcolor}
% If you use beamer only pass "xcolor=table" option, i.e. \documentclass[xcolor=table]{beamer}
\begin{table}[ht!]
    \centering
    \resizebox{\textwidth}{!}{%
        \begin{tabular}{|l|
                >{\columncolor[HTML]{34FF34}}c |
                >{\columncolor[HTML]{F8A102}}c |}
            \hline
            \multicolumn{1}{|c|}{\cellcolor[HTML]{EFEFEF}\textbf{Característica}} & \cellcolor[HTML]{EFEFEF}\textbf{Software Switch OvS} & \cellcolor[HTML]{EFEFEF}\textbf{Software Switch BOFUSS}        \\ \hline
            Solidez                                                               & Más solido dado que tienen una comunidad más grande  & {\color[HTML]{333333} Sistema de test inexistente}             \\ \hline
            Popularidad                                                           & Ampliamente utilizado en entornos de virtualización  & Utilizado por la academia para hacer pruebas de concepto       \\ \hline
            Rendimiento                                                           & Alto rendimineto                                     & Pobre, corre en espacio de usuario                             \\ \hline
            Depuración                                                            & \cellcolor[HTML]{F8A102}Difícil, corre en el Kernel  & \cellcolor[HTML]{34FF34}Asequible, corre en espacio de usuario \\ \hline
            Facilidad para añadir código                                          & \cellcolor[HTML]{F8A102}Difícil, corre en el Kernel  & \cellcolor[HTML]{34FF34}Relativamente sencillo                 \\ \hline
            Versatilidad                                                          & Versátil, puede trabajar en standalone               & Requiere de un agente de control                               \\ \hline
            Curva de aprendizaje                                                  & \cellcolor[HTML]{F8A102}Muy lenta                    & \cellcolor[HTML]{34FF34}Muy rápida                             \\ \hline
        \end{tabular}%
    }
    \caption{Comparativa del \glsentryshort{ovs} con el  \glsentryshort{bofus}}
    \label{tab:switches}
\end{table}


Después de analizar las fortalezas y debilidades de cada opción, ver tabla \ref{tab:switches}, hemos llegado a una decisión sobre qué controlador utilizar en nuestro proyecto. Tanto \gls{ovs} como \gls{bofus} tienen características distintivas que deben ser consideradas en nuestro caso de uso. Pero finalmente se ha elegido \gls{bofus} como \textit{software switch} \gls{sdn}, dado que para nuestro proyecto es la opción más adecuada por haber una implementación preliminar de in-band\footnote{\url{https://github.com/NETSERV-UAH/in-BOFUSS}}.