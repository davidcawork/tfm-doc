\section{Agente de control \glsentryshort{sdn}}
\label{sec:ana_controller}

En este punto se tiene que valorar qué agente de control \gls{sdn}, es decir controlador \gls{sdn} se va a utilizar. Se tendrán en cuenta las condiciones del entorno sobre el cual se va a desplegar, así como las características propias de cada controlador, así como la facilidad y flexibilidad que nos entregue cada uno para desplegarlo sobre la plataforma emulada. Las opciones que se han considerado para este cometido son las siguientes:

\begin{itemize}
    \item Ryu, explicado anteriormente en la Sección \ref{subsec:ryu} del estado del arte.

    \item \gls{onos}, explicado anteriormente en la Sección \ref{subsec:ONOS} del estado del arte.
\end{itemize}

Como se ha mencionado previamente, cada herramienta presenta sus fortalezas y debilidades. Por lo tanto, llevar a cabo una comparativa es pertinente para nuestro caso de uso, a fin de determinar cuál de las dos opciones resulta más conveniente para este proyecto.

\begin{itemize}
    \item El controlador ONOS exhibe una mayor potencia en comparación con Ryu, siendo ampliamente utilizado por los operadores de red debido a su destacado rendimiento y solidez.
    \item El controlador ONOS supera a Ryu en términos de procesamiento de paquetes, evidenciando un rendimiento superior en este aspecto.
    \item Por otro lado, el controlador Ryu presenta una menor carga y es más sencillo de depurar y ampliar con nuevas funcionalidades en caso de ser necesario.
    \item Asimismo, debido a su menor tamaño, el controlador Ryu se despliega e instala de manera más rápida en comparación con el controlador ONOS.
    \item La curva de aprendizaje de Ryu es también más pequeña en comparación con ONOS, dado su tamaño reducido.
    \item Aunque una ventaja potencial de ONOS es su aplicación de descubrimiento topológico, que cuenta con una interfaz web bastante \textit{fancy}, esta aplicación utiliza el protocolo \gls{lldp} para el descubrimiento. Sin embargo, dicho protocolo no permite descubrir la configuración de enlaces inalámbricos, sino solo los nodos, lo cual lo hace inútil en este contexto. Aunque existen algunas publicaciones \cite{martinez2021ehddp} que abordan esta necesidad, resultaría inviable implementar dicho protocolo en el proyecto, ya que excedería los objetivos temporales y de alcance establecidos.
\end{itemize}


% Please add the following required packages to your document preamble:
% \usepackage{graphicx}
% \usepackage[table,xcdraw]{xcolor}
% If you use beamer only pass "xcolor=table" option, i.e. \documentclass[xcolor=table]{beamer}
\begin{table}[ht!]
    \centering
    \resizebox{\textwidth}{!}{%
        \begin{tabular}{|l|
                >{\columncolor[HTML]{F8A102}}c |
                >{\columncolor[HTML]{34FF34}}c |}
            \hline
            \multicolumn{1}{|c|}{\cellcolor[HTML]{EFEFEF}\textbf{Característica}} & \cellcolor[HTML]{EFEFEF}\textbf{Controlador ONOS}                           & \cellcolor[HTML]{EFEFEF}\textbf{Controlador Ryu}                                 \\ \hline
            Solidez                                                               & \cellcolor[HTML]{34FF34}Más solido dado que tienen una comunidad más grande & \cellcolor[HTML]{F8A102}{\color[HTML]{333333} Sistema de test bastante pobre}    \\ \hline
            Popularidad                                                           & \cellcolor[HTML]{34FF34}Ampliamente utilizado por los operadores de red     & \cellcolor[HTML]{F8A102}Utilizado por la academia para hacer pruebas de concepto \\ \hline
            Rendimiento                                                           & \cellcolor[HTML]{34FF34}Alto rendimineto                                    & \cellcolor[HTML]{F8A102}Es un script de Python, lenguaje interpretado            \\ \hline
            Depuración                                                            & Al ser tan grande es difícil depurarlo                                      & Es un script de Python, es fácil de depurar                                      \\ \hline
            Facilidad para añadir código                                          & Añadir código al core es muy complicado                                     & Añadir código es muy sencillo, solo hay que añadir las funciones nuevas          \\ \hline
            Tamaño                                                                & Es muy grande                                                               & Relativamente ligero                                                             \\ \hline
            Despliegue e Instalación                                              & Al ser tan grande, tarda mucho                                              & Rápido                                                                           \\ \hline
            Curva de aprendizaje                                                  & Muy lenta                                                                   & Muy rápida                                                                       \\ \hline
            Interfaz de usuario                                                   & \cellcolor[HTML]{34FF34}Refinada y cercana al usuario                       & \cellcolor[HTML]{F8A102}Bastante pobre                                           \\ \hline
        \end{tabular}%
    }
    \caption{Comparativa del controlador \glsentryshort{onos}  con el controlador Ryu}
    \label{tab:controllers}
\end{table}

Después de analizar las fortalezas y debilidades de cada opción, ver tabla \ref{tab:controllers}, hemos llegado a una decisión sobre qué controlador utilizar en nuestro proyecto. Tanto \gls{onos} como Ryu tienen características distintivas que deben ser consideradas en nuestro caso de uso. Pero finalmente se ha elegido Ryu como controlador \gls{sdn}, dado que para nuestro proyecto es la opción más adecuada.