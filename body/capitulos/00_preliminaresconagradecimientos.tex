%%%%%%%%%%%%%%%%%%%%%%%%%%%%%%%%%%%%%%%%%%%%%%%%%%%%%%%%%%%%%%%%%%%%%%%%
% Plantilla TFG/TFM
% Escuela Politécnica Superior de la Universidad de Alicante
% Realizado por: Jose Manuel Requena Plens
% Contacto: info@jmrplens.com / Telegram:@jmrplens
%%%%%%%%%%%%%%%%%%%%%%%%%%%%%%%%%%%%%%%%%%%%%%%%%%%%%%%%%%%%%%%%%%%%%%%%

%%%%%%%%%%%% Dedicatoria %%%%%%%%%%%%%%%%%%%%%%%%%%%%%%%%%%%%%%%%%%%%%%%%

\cleardoublepage %salta a nueva página impar
% Aquí va la dedicatoria si la hubiese. Si no, comentar la(s) linea(s) siguientes
\chapter*{}
\setlength{\leftmargin}{0.5\textwidth}
\setlength{\parsep}{0cm}
\addtolength{\topsep}{0.5cm}
\begin{flushright}
	\small\em{
		A mis hermanas, Natalia y Violeta,\\
		quienes día a día, por oscura que sea la noche,\\
		arrojan luz y esperanza a mi vida.
	}
\end{flushright}


%%%%%%%%%%%%%%%%%%%%%%%%%%%%%%%%%%%%%%%%%%%%%%%%%%%%%%%%%%%%%%%%%%%%%%%%

%%%%%%%%%%%% Agradecimientos %%%%%%%%%%%%%%%%%%%%%%%%%%%%%%%%%%%%%%%%%%%

\chapter*{Agradecimientos}

\thispagestyle{empty}
\vspace{1cm}

Quiero empezar agradeciendo y reconociendo a mi tutora, Elisa Rojas, sin la cual este trabajo no habría sido posible. Quien desde segundo de carrera creyó en mi, y aun día de hoy, sigue apostando día a día en mis capacidades, incluso cuando ni yo mismo soy capaz de verlas. Su destreza y conocimiento, su apoyo incondicional y carisma, su maestría y pasión por lo que hace, y a lo que se dedica, han hecho que etapa, tras etapa académica siga aprendiendo y disfrutando como el primer día. Este trabajo ha sido financiado por subvenciones de la Comunidad de Madrid a través de los proyectos TAPIR-CM (S2018/TCS-4496) y MistLETOE-CM (CM/JIN/2021-006), y por el proyecto ONENESS (PID2020-116361RA-I00) del Ministerio de Ciencia e Innovación de España.\newline

También me gustaría agradecer a mi familia, por su cariño, comprensión e inpiración en estos meses que han sido tan duros para mi. 	Y que decir de mis amigos, a los \textit{Caye de Calle}, a los \textit{C de Chill},  a mis estimados \textit{Pueblerinos}, como no, a Pablo y Olga, a mis queridas Noci, a mi señor abuelo de confianza, Boby, a la señorita Laura de Diego, mis compis de la uni y toda la gente nueva que ha llegado a mi vida durante estos meses, a todos vosotros, gracias por las risas y los buenos momentos que hemos compartido juntos. Gracias de verdad. \newline


No puedo terminar sin agradecer a toda la gente del Laboratorio LE34, quienes me alentado a seguir por este arduo camino de la investigación y quienes con sus consejos y experiencias han ido conformando al ingeniero que soy a día de hoy.

\vspace{0.5cm}

Sinceramente, mil gracias a todos.




\cleardoublepage %salta a nueva página impar

%%%%%%%%%%%%%%%%%%%%%%%%%%%%%%%%%%%%%%%%%%%%%%%%%%%%%%%%%%%%%%%%%%%%%%%%


%%%%%%%%%%%%  Resumen corto  %%%%%%%%%%%%%%%%%%%%%%%%%%%%%%%%%%%%%%%%%%%
\chapter{Resumen}
\thispagestyle{empty}
% Resumen de 100 palabras: Hay 110 actualmente
Este \gls{tfg} se enfoca en el estudio de las distintas tecnologías disponibles para la definición del denominado \textit{datapath}, con el objetivo de conseguir una integración de dispositivos \gls{iot} en entornos de \gls{sdn}. \newline

Las dos principales tecnologías que se han analizado para la definición de datapaths son \gls{xdp} y el lenguaje P4. A su vez, se han planteado casos de uso donde se quiere demostrar qué ciertas funcionalidades, que debe tener un nodo \gls{iot}, pueden ser implementadas. Cuando los casos de uso se realicen, se justificará qué tecnología es más viable en la definición de los datapaths para alcanzar la integración de nodos \gls{iot} en entornos \gls{sdn}.

\vspace{1cm}

\textbf{Palabras clave}: \href{https://scholar.google.es/scholar?q=Internet+of+Things}{IoT}; \href{https://www.opennetworking.org/sdn-definition}{SDN};
\href{https://p4.org/}{Lenguaje P4}; \href{https://scholar.google.es/scholar?q=XDP+linux}{XDP};
\href{https://scholar.google.es/scholar?q=Datapaths}{Plano de datos}


\cleardoublepage %salta a nueva página impar

%%%%%%%%%%%%%%%%%%%%%%%%%%%%%%%%%%%%%%%%%%%%%%%%%%%%%%%%%%%%%%%%%%%%%%%%


%%%%%%%%%%%%  Resumen corto - Inglés %%%%%%%%%%%%%%%%%%%%%%%%%%%%%%%%%%%
\chapter{Abstract}
\thispagestyle{empty}

This Bachelor's Degree Final Project is focused on the study of the different technologies available for the definition of the so-called datapaths, with the aim of achieving an integration of Internet of Things (\gls{iot}) devices in Software-Defined Networking (\gls{sdn}) environments. \newline

The two main technologies that have been analyzed for the definition of datapaths are
eXpress Data Path (\gls{xdp}) and the P4 language. At the same time, a set of use cases has been presented to demonstrate that certain functionalities that an \gls{iot} node must have can be implemented. When the use cases are carried out, it will be justified which technology is more viable in the definition of the datapaths to achieve the integration of \gls{iot} nodes in \gls{sdn} environments.

\vspace{1cm}

\textbf{Keywords}: \href{https://scholar.google.es/scholar?q=Internet+of+Things}{IoT}; \href{https://www.opennetworking.org/sdn-definition}{SDN};
\href{https://p4.org/}{P4 Language}; \href{https://scholar.google.es/scholar?q=XDP+linux}{XDP};
\href{https://scholar.google.es/scholar?q=Datapaths}{Datapaths}



\cleardoublepage %salta a nueva página impar

%%%%%%%%%%%%%%%%%%%%%%%%%%%%%%%%%%%%%%%%%%%%%%%%%%%%%%%%%%%%%%%%%%%%%%%%


%%%%%%%%%%%%  Resumen extendido     %%%%%%%%%%%%%%%%%%%%%%%%%%%%%%%%%%%

%%%%%%%%%%%%%%%%%%%%%%%%%%%%  Cita   %%%%%%%%%%%%%%%%%%%%%%%%%%%%%%%%%%%
% Aquí va la cita célebre si la hubiese. Si no, comentar la(s) linea(s) siguientes
\chapter*{}
\setlength{\leftmargin}{0.5\textwidth}
\setlength{\parsep}{0cm}
\addtolength{\topsep}{0.5cm}
\begin{flushright}
	\small\em{
		``No hay ningún viento favorable para el que no sabe a que puerto se dirige"
	}
\end{flushright}
\begin{flushright}
	\small{
		Arthur Schopenhauer.
	}
\end{flushright}
\cleardoublepage %salta a nueva página impar
%%%%%%%%%%%%%%%%%%%%%%%%%%%%%%%%%%%%%%%%%%%%%%%%%%%%%%%%%%%%%%%%%%%%%%%%