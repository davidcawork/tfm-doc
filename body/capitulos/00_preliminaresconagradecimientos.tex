%%%%%%%%%%%%%%%%%%%%%%%%%%%%%%%%%%%%%%%%%%%%%%%%%%%%%%%%%%%%%%%%%%%%%%%%
% Plantilla TFG/TFM
% Escuela Politécnica Superior de la Universidad de Alicante
% Realizado por: Jose Manuel Requena Plens
% Contacto: info@jmrplens.com / Telegram:@jmrplens
%%%%%%%%%%%%%%%%%%%%%%%%%%%%%%%%%%%%%%%%%%%%%%%%%%%%%%%%%%%%%%%%%%%%%%%%

%%%%%%%%%%%% Dedicatoria %%%%%%%%%%%%%%%%%%%%%%%%%%%%%%%%%%%%%%%%%%%%%%%%

\cleardoublepage %salta a nueva página impar
% Aquí va la dedicatoria si la hubiese. Si no, comentar la(s) linea(s) siguientes
\chapter*{}
\setlength{\leftmargin}{0.5\textwidth}
\setlength{\parsep}{0cm}
\addtolength{\topsep}{0.5cm}
\begin{flushright}
	\small\em{
		A mis hermanas, Natalia y Violeta,\\
		quienes día a día, por oscura que sea la noche,\\
		arrojan luz y esperanza a mi vida.
	}
\end{flushright}


%%%%%%%%%%%%%%%%%%%%%%%%%%%%%%%%%%%%%%%%%%%%%%%%%%%%%%%%%%%%%%%%%%%%%%%%

%%%%%%%%%%%% Agradecimientos %%%%%%%%%%%%%%%%%%%%%%%%%%%%%%%%%%%%%%%%%%%

\chapter*{Agradecimientos}

\thispagestyle{empty}
\vspace{1cm}

Quiero empezar agradeciendo y reconociendo a mi tutora, Elisa Rojas, sin la cual este trabajo no habría sido posible. Quien desde segundo de carrera creyó en mi, y aun día de hoy, sigue apostando día a día en mis capacidades, incluso cuando ni yo mismo soy capaz de verlas. Su destreza y conocimiento, su apoyo incondicional y carisma, su maestría y pasión por lo que hace, y a lo que se dedica, han hecho que etapa, tras etapa académica siga aprendiendo y disfrutando como el primer día. Este trabajo ha sido financiado por subvenciones de la Comunidad de Madrid a través de los proyectos TAPIR-CM (S2018/TCS-4496) y MistLETOE-CM (CM/JIN/2021-006), y por el proyecto ONENESS (PID2020-116361RA-I00) del Ministerio de Ciencia e Innovación de España.\newline

También me gustaría agradecer a mi familia, por su cariño, comprensión e inpiración en estos meses que han sido tan duros para mi. 	Y que decir de mis amigos, a los \textit{Caye de Calle}, a los \textit{C de Chill},  a mis estimados \textit{Pueblerinos}, como no, a Pablo y Olga, a mis queridas Noci, a mi señor abuelo de confianza, Boby, a la señorita Laura de Diego, mis compis de la uni y toda la gente nueva que ha llegado a mi vida durante estos meses, a todos vosotros, gracias por las risas y los buenos momentos que hemos compartido juntos. Gracias de verdad. \newline


No puedo terminar sin agradecer a toda la gente del Laboratorio LE34, quienes me alentado a seguir por este arduo camino de la investigación y quienes con sus consejos y experiencias han ido conformando al ingeniero que soy a día de hoy.

\vspace{0.5cm}

Sinceramente, mil gracias a todos.




\cleardoublepage %salta a nueva página impar

%%%%%%%%%%%%%%%%%%%%%%%%%%%%%%%%%%%%%%%%%%%%%%%%%%%%%%%%%%%%%%%%%%%%%%%%


%%%%%%%%%%%%  Resumen corto  %%%%%%%%%%%%%%%%%%%%%%%%%%%%%%%%%%%%%%%%%%%
\chapter{Resumen}
\thispagestyle{empty}
% Resumen de 100 palabras: Hay 110 actualmente
En este \gls{tfm} se presenta el diseño e implementación de un protocolo de control escalable de redes \gls{iot} para entornos \gls{sdn} en la nueva generación de redes moviles, \gls{6g}. Dicho protocolo de control seguirá un paradigma de control de tipo \textit{in-band}, con el cual se dotará de conectividad a los nodos de la red con el ente de control, empleando el plano de datos para la transmisión de información de control.\\
\\
En arás de completar el proyecto,  se ha partido por analizar las necesidades y caracteristicas de las distintas tecnologías que se emplearán en ejecución del objetivos predefinidos y así discernir aquellas herramientas necesarias para la implementación del protocolo control. Una vez seleccionadas las herramientas, se estudiarán a fondo para realizar una implementación lo optimizada en la medida de lo posible. Este proyecto concluirá con la validación mediante emulación del protocolo desarrollado para comprobar el correcto funcionamiento del mismo en distintos casos de uso.

\vspace{1cm}

\textbf{Palabras clave}: \href{https://scholar.google.com/scholar?q=6g}{6G}; \href{https://scholar.google.es/scholar?q=Internet+of+Things}{IoT};
\href{https://www.opennetworking.org/sdn-definition}{SDN}; \href{https://scholar.google.com/scholar?q=sdn+in-band+control}{Control in-band};
\href{https://scholar.google.es/scholar?q=control+plane+sdn}{Plano de control}


\cleardoublepage %salta a nueva página impar

%%%%%%%%%%%%%%%%%%%%%%%%%%%%%%%%%%%%%%%%%%%%%%%%%%%%%%%%%%%%%%%%%%%%%%%%


%%%%%%%%%%%%  Resumen corto - Inglés %%%%%%%%%%%%%%%%%%%%%%%%%%%%%%%%%%%
\chapter{Abstract}
\thispagestyle{empty}

In this Master's Thesis (TFM) we present the design and implementation of a scalable control protocol for Internet of Things (IoT) networks for Software-Defined Networking (SDN) environments in the new generation of mobile networks, the sixth generation of mobile technologies (6G). This control protocol will be based on an \textit{in-band} control paradigm, which will provide connectivity between the network nodes and the control entity, using the data plane for the transmission of control information.\\
\\
In order to fulfil the project, we have started by analysing the requirements and characteristics of the different technologies that will be used in the execution of the predefined objectives and thus be able to determine the tools necessary for the implementation of the control protocol. Once the tools have been selected, they will be studied in depth in order to carry out an optimised implementation as far as possible. This project will conclude with the validation the developed protocol by means of emulation to check its correct operation in different use cases.


\vspace{1cm}

\textbf{Keywords}: \href{https://scholar.google.com/scholar?q=6g}{6G}; \href{https://scholar.google.es/scholar?q=Internet+of+Things}{IoT};
\href{https://www.opennetworking.org/sdn-definition}{SDN}; \href{https://scholar.google.com/scholar?q=sdn+in-band+control}{In-band control};
\href{https://scholar.google.es/scholar?q=control+plane+sdn}{Control plane}



\cleardoublepage %salta a nueva página impar

%%%%%%%%%%%%%%%%%%%%%%%%%%%%%%%%%%%%%%%%%%%%%%%%%%%%%%%%%%%%%%%%%%%%%%%%


%%%%%%%%%%%%  Resumen extendido     %%%%%%%%%%%%%%%%%%%%%%%%%%%%%%%%%%%

%%%%%%%%%%%%%%%%%%%%%%%%%%%%  Cita   %%%%%%%%%%%%%%%%%%%%%%%%%%%%%%%%%%%
% Aquí va la cita célebre si la hubiese. Si no, comentar la(s) linea(s) siguientes
\chapter*{}
\setlength{\leftmargin}{0.5\textwidth}
\setlength{\parsep}{0cm}
\addtolength{\topsep}{0.5cm}
\begin{flushright}
	\small\em{
		``No hay ningún viento favorable para el que no sabe a que puerto se dirige"
	}
\end{flushright}
\begin{flushright}
	\small{
		Arthur Schopenhauer.
	}
\end{flushright}
\cleardoublepage %salta a nueva página impar
%%%%%%%%%%%%%%%%%%%%%%%%%%%%%%%%%%%%%%%%%%%%%%%%%%%%%%%%%%%%%%%%%%%%%%%%