%%%%%%%%%%%%%%%%%%%%%%%%%%%%%%%%%%%%%%%%%%%%%%%%%%%%%%%%%%%%%%%%%%%%%%%%
% Plantilla TFG/TFM
% Escuela Politécnica Superior de la Universidad de Alicante
% Realizado por: Jose Manuel Requena Plens
% Contacto: info@jmrplens.com / Telegram:@jmrplens
%%%%%%%%%%%%%%%%%%%%%%%%%%%%%%%%%%%%%%%%%%%%%%%%%%%%%%%%%%%%%%%%%%%%%%%%

%%%%%%%%%%%% Dedicatoria %%%%%%%%%%%%%%%%%%%%%%%%%%%%%%%%%%%%%%%%%%%%%%%%

\cleardoublepage %salta a nueva página impar
% Aquí va la dedicatoria si la hubiese. Si no, comentar la(s) linea(s) siguientes
\chapter*{}
\setlength{\leftmargin}{0.5\textwidth}
\setlength{\parsep}{0cm}
\addtolength{\topsep}{0.5cm}
\begin{flushright}
	\small\em{
		A mis hermanas, Natalia y Violeta,\\
		quienes día a día, por oscura que sea la noche,\\
		arrojan luz y esperanza a mi vida.
	}
\end{flushright}


%%%%%%%%%%%%%%%%%%%%%%%%%%%%%%%%%%%%%%%%%%%%%%%%%%%%%%%%%%%%%%%%%%%%%%%%

%%%%%%%%%%%% Agradecimientos %%%%%%%%%%%%%%%%%%%%%%%%%%%%%%%%%%%%%%%%%%%

\chapter*{Agradecimientos}

\thispagestyle{empty}
\vspace{1cm}

Este trabajo no habría sido posible sin el apoyo y el estímulo de mis tutores, Elisa y Joaquín, que compartieron su destreza y conocimiento desde el primer día que les conocí. Quienes, bajo su supervisión, me han enseñado lo entretenido y bonito que es el mundo de las Redes. \newline

También me gustaría agradecer a mi profesor de Sistemas Operativos, Juan Ignacio García Tejedor, cuya maestría y pasión por lo que hace, y a lo que se dedica, me hicieron aprender y disfrutar cada día que iba sus clases. Es más, aún habiendo terminado la asignatura me ayudó en uno de los momentos más difíciles de este trabajo, arrojando luz y aplanando el duro camino al Kernel de Linux. No me podía olvidar de mis amigos, Rubén y Laura, con los que he compartido momentos de estrés y tardes de risas, y de Pablo, a quien lo he conocido casi al final de la carrera, pero con el que comparto un montón de ratos y pasiones. \newline

No puedo terminar sin agradecer a mis compañeros del Laboratorio LE34, quienes me han enseñado a ver los problemas con perspectiva, en ocasiones, a priorizar según el caso. Me han alegrado cada semana de estos dos últimos años con su apoyo y compañía, y  me han alentado a mejorar cada día, marcándome el camino en un futuro incierto.

\vspace{0.5cm}

Sinceramente, mil gracias a todos.




\cleardoublepage %salta a nueva página impar

%%%%%%%%%%%%%%%%%%%%%%%%%%%%%%%%%%%%%%%%%%%%%%%%%%%%%%%%%%%%%%%%%%%%%%%%


%%%%%%%%%%%%  Resumen corto  %%%%%%%%%%%%%%%%%%%%%%%%%%%%%%%%%%%%%%%%%%%
\chapter{Resumen}
\thispagestyle{empty}
% Resumen de 100 palabras: Hay 110 actualmente
Este \gls{tfg} se enfoca en el estudio de las distintas tecnologías disponibles para la definición del denominado \textit{datapath}, con el objetivo de conseguir una integración de dispositivos \gls{iot} en entornos de \gls{sdn}. \newline

Las dos principales tecnologías que se han analizado para la definición de datapaths son \gls{xdp} y el lenguaje P4. A su vez, se han planteado casos de uso donde se quiere demostrar qué ciertas funcionalidades, que debe tener un nodo \gls{iot}, pueden ser implementadas. Cuando los casos de uso se realicen, se justificará qué tecnología es más viable en la definición de los datapaths para alcanzar la integración de nodos \gls{iot} en entornos \gls{sdn}.

\vspace{1cm}

\textbf{Palabras clave}: \href{https://scholar.google.es/scholar?q=Internet+of+Things}{IoT}; \href{https://www.opennetworking.org/sdn-definition}{SDN};
\href{https://p4.org/}{Lenguaje P4}; \href{https://scholar.google.es/scholar?q=XDP+linux}{XDP};
\href{https://scholar.google.es/scholar?q=Datapaths}{Plano de datos}


\cleardoublepage %salta a nueva página impar

%%%%%%%%%%%%%%%%%%%%%%%%%%%%%%%%%%%%%%%%%%%%%%%%%%%%%%%%%%%%%%%%%%%%%%%%


%%%%%%%%%%%%  Resumen corto - Inglés %%%%%%%%%%%%%%%%%%%%%%%%%%%%%%%%%%%
\chapter{Abstract}
\thispagestyle{empty}

This Bachelor's Degree Final Project is focused on the study of the different technologies available for the definition of the so-called datapaths, with the aim of achieving an integration of Internet of Things (\gls{iot}) devices in Software-Defined Networking (\gls{sdn}) environments. \newline

The two main technologies that have been analyzed for the definition of datapaths are
eXpress Data Path (\gls{xdp}) and the P4 language. At the same time, a set of use cases has been presented to demonstrate that certain functionalities that an \gls{iot} node must have can be implemented. When the use cases are carried out, it will be justified which technology is more viable in the definition of the datapaths to achieve the integration of \gls{iot} nodes in \gls{sdn} environments.

\vspace{1cm}

\textbf{Keywords}: \href{https://scholar.google.es/scholar?q=Internet+of+Things}{IoT}; \href{https://www.opennetworking.org/sdn-definition}{SDN};
\href{https://p4.org/}{P4 Language}; \href{https://scholar.google.es/scholar?q=XDP+linux}{XDP};
\href{https://scholar.google.es/scholar?q=Datapaths}{Datapaths}



\cleardoublepage %salta a nueva página impar

%%%%%%%%%%%%%%%%%%%%%%%%%%%%%%%%%%%%%%%%%%%%%%%%%%%%%%%%%%%%%%%%%%%%%%%%


%%%%%%%%%%%%  Resumen extendido     %%%%%%%%%%%%%%%%%%%%%%%%%%%%%%%%%%%

%%%%%%%%%%%%%%%%%%%%%%%%%%%%  Cita   %%%%%%%%%%%%%%%%%%%%%%%%%%%%%%%%%%%
% Aquí va la cita célebre si la hubiese. Si no, comentar la(s) linea(s) siguientes
\chapter*{}
\setlength{\leftmargin}{0.5\textwidth}
\setlength{\parsep}{0cm}
\addtolength{\topsep}{0.5cm}
\begin{flushright}
	\small\em{
		``No hay ningún viento favorable para el que no sabe a que puerto se dirige"
	}
\end{flushright}
\begin{flushright}
	\small{
		Arthur Schopenhauer.
	}
\end{flushright}
\cleardoublepage %salta a nueva página impar
%%%%%%%%%%%%%%%%%%%%%%%%%%%%%%%%%%%%%%%%%%%%%%%%%%%%%%%%%%%%%%%%%%%%%%%%