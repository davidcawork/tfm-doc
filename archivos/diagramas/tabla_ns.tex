\begin{table}[ht]
\centering
\resizebox{\textwidth}{!}{%
\begin{tabular}{|l|l|}
\hline
\rowcolor[HTML]{EFEFEF} 
\multicolumn{1}{|c|}{\cellcolor[HTML]{EFEFEF}{\color[HTML]{24292E} \textbf{Tipo de Namespace}}} & \multicolumn{1}{c|}{\cellcolor[HTML]{EFEFEF}{\color[HTML]{24292E} \textbf{Descripción}}}                                                                                                                                                     \\ \hline
\textbf{Cgroup}                                                                                          & \begin{tabular}[c]{@{}l@{}}Namespace utilizada generalmente para establecer unos limites de recursos,  por ejemplo, \\ CPU, memoria, lecturas y escrituras a disco, de todos los procesos que corran dentro de dicha Namespace.\end{tabular} \\ \hline
Time                                                                                            & Namespace para establecer una hora del sistema diferente a la del sistema.                                                                                                                                                                   \\ \hline
\textbf{Network}                                                                                         & Namespace utilizada para tener una replica aislada del stack de red del sistema, dentro del propio sistema.                                                                                                                                  \\ \hline
\textbf{User}                                                                                            & Namespace utilizada para tener aislados a un grupo de usuarios.                                                                                                                                                                              \\ \hline
\textbf{PID}                                                                                             & Namespace utilizada para tener identificadores de proceso independientes de otras namespaces.                                                                                                                                                \\ \hline
\textbf{IPC}                                                                                             & Namespace utilizada para aislar los mecanismos de comunicación entre procesos.                                                                                                                                                               \\ \hline
\textbf{Uts}                                                                                             & Namespace utilizada para establecer un nombre de Host y nombre de dominio diferentes de los establecidos en el sistema                                                                                                                       \\ \hline
\textbf{Mount}                                                                                           & Namespace utilizada para aislar los puntos de montaje en el sistema de archivos.                                                                                                                                                             \\ \hline
\end{tabular}%
}
\caption{Resumen de los tipos de Namespaces en el Kernel de Linux}
\label{tab:linux_ns}
\end{table}